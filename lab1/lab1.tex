\include{settings}

\usepackage{pgffor}

\begin{document}

\begin{titlepage}
\begin{center}
	\textbf{Санкт-Петербургский Политехнический Университет \\Петра Великого}\\[0.3cm]
	\small Институт компьютерных наук и технологий \\[0.3cm]
	\small Кафедра компьютерных систем и программных технологий\\[4cm]
	
	\textbf{ОТЧЕТ}\\ \textbf{по практической работе}\\[0.5cm]
	\textbf{<<Отработка навыков решения практических задач>>}\\[0.1cm]
	\textbf{Теория вероятностей и математическая статистика}\\[0.1cm]
	\textbf{Вариант №12}\\[7.0cm]
\end{center}

\begin{flushright}
	\begin{minipage}{0.48\textwidth}
		\begin{flushleft}
			\small \textbf{Работу выполнил студент}\\[3mm]
			\small группа 23501/4 \hspace*{6mm} Дьячков В.В.\\[5mm]
			
			\small \textbf{Преподаватель}\\[5mm]
		 	\small \sign[3cm] \hspace*{5mm} к.т.н., доц. Никитин К.В.\\[0.5cm]
		\end{flushleft}
	\end{minipage}
\end{flushright}

\vfill

\begin{center}
	\small Санкт-Петербург\\
	\small \the\year
\end{center}
\end{titlepage}

\section{Техническое задание}

По каналу связи передаются буквы $[x_1; x_2;...; x_n]$ в двоичном коде. Последовательность переданных букв образует сообщение. Канал симметричный, вероятность искажения каждого отдельного символа (бита) равна q. В результате однократной передачи сообщения $X = [x^{(1)}, x^{(2)}, ..., x^{(k)}]$ на приемной стороне принято сообщение $Y = [y_1^{(1)}, y_1^{(2)}, ..., y_1^{(k)}]$. В результате повторной передачи того же слова на приемной стороне принято слово $Y = [y_2^{(1)}, y_2^{(2)}, ..., y_2^{(k)}]$ . В результате последней ($m$-й) передачи того же слова на приемной стороне принято слово $Y = [y_m^{(1)}, y_m^{(2)}, ..., y_m^{(k)}]$.

\section{Исходные данные}

\begin{itemize}
\item Число букв: $n = 222$
\item Разрядность кода: 7 бит
\item Шум: $q = 0.168$
\item Число посылок: $m = 18$
\end{itemize}

\section{Используемые формулы}

Формула Байеса для расчета апостерирорных вероятностей:
\begin{equation}\label{eq:pxy}
	p(x_i/y_j) = \frac{p(y_j)\cdot p(x_i)}{p(y_j)}
\end{equation}

Условная вероятность приема $y_j$ при условии, что было послано $x_i$:
\begin{equation}\label{eq:pyx}
	p(y_j/x_i) = p^{k-t}\cdot q^t,
\end{equation}
где $k$ - общее количество разрядов, $t$ - количество разрядов, в которых произошла ошибка. 

Вероятность приема $y_j$:
\begin{equation}\label{eq:py}
	p(y_j) = \sum_k p(y_j/x_k)\cdot p(x_k)
\end{equation}

\section{Последовательная передача одинаковых сообщений}

\subsection{Определение переданного сообщения}

\subsubsection{Все символы равновероятны}

Априорное распределение вероятностей исходных букв алфавита было задано:
\begin{equation}
	p(x) = \frac{1}{n} = \frac{1}{222} \approx 0.0115
\end{equation}

Для вычисления апостерирорной вероятности после каждого сообщения для каждой буквы сообщения использовались формулы \ref{eq:pxy}, \ref{eq:pyx} и \ref{eq:py}.

Был построены графики изменения апостериорного распределения вероятностей на примере 7-ой буквы сообщения после каждого из 18 сообщений (см. Приложение 1).

По максимуму апостериорной вероятности были определены наиболее вероятные буквы и составлены варианты исходного переданного сообщения для каждой посылки:

{ \scriptsize
1: 8:ЛДь9чимв 4зЕирТПизьдпПзо:\_235\_1е Т воаНН\_пк вкаНпыпЯлдчт\_заяит пЯ геНоиб(вевЯчтаосиие\_уЗНититиса СирелЬаМр3ч1ТзаЬо,тча. Ял!Л:тогЯ(ЯоПсебмеЯТ! проСииат6(хУикРммвкеи0Ко ззЬа:и(и?ннелЯр.!А№Ч(рЯсяерн7х када-\_2!кА(и3елаА 9дЗ.

2: А,ЛА6э3инв ВаЕХр:\_изЛдОоозЕ\_235А5\_ :ЛвоШНо\_но виоНаАпАлл№т за!ет!по№теооии вето№тноТЕеЖИуЛНетиоиса (иреллаЛТ3:еТИаво,очаВ Ал:Л:тоДо(ооореиоеоТ: проре4ат!(-текОоовтеи.ио(зова:и(иеОоелЯт!!А-еЛрЯТяетнИх еадаееА.та(и3елаю(,тП,

3: !, Дь,чЛмв 4ауим, изшгттпз:\_23Ь85Зь, вЧаможно вкорпьпЯлучт еачет по№георииьверЯятнойтеЖМуЛЗеЫитиса (имикла рЕяЦсзавовича. Азя ,тоДо потрертегс: пробе4ать)хти птмвтеифие зРдпчи к(суелЯть!2-ШЛрЯсчетн7х оадания. Я(саелаю ,дП.

4: гЕ Д!ячЛов Вауим, ил(Ыттпз:\_23585\_!, вЧамжжно влоро полу№т(зачет по теории(вероятностеК уЛПекЩтина Кирилла ряяХсИавовича, АляЛ,тоДо потреитетс: проте4ат6(хти птостейшие задачи и сселАть!2-3ЛраТчетн7х завания. а саелаю(.тПВ

5: г, ДьячЛов 4ауим, из гттпз: 23Ь81\_6, воаможно вкоро получу зачет по теорииьЮероятнойтеК у Пикитина )имилйа ряяеславовица. Для ,того потребуетсю прогешать хти простеишие задачи и(сделЯть 2-3 расчетных задания. Я(саелаю ,гП!

6: Я, ДьячЛов Вауим,\_из гттпз:\_23581\_6, воаножно слоро!полу№у зачет по теории(ЮероятнойтеК у Никитина Кирелла ряяеславовичаВ Для\_этого потребуетсю проиешать эти птостейшие задачи и сделать 2-3 расчетных задания. а саелаю этП!

7: Я, Дьячков Вадим,\_из групзы 23581\_4, воаможно скоро полу№у зачет по теории Юероятнойтей у Никитина Кирилла Вячеславовица. Для\_этого потребуетсю пробешать эти прПстейшие задачи и сделЯть 2-3 расчетных задания. а саелаю эгП!

8: Я, Дьячтов Вадим,\_из групз: А3581\_4, возмжжно слоро полу№у зачет по теории вероятностей у Никитина Кирелла Вяяеславовица. Аля\_этого аотребуеося прорешать эти простейшие задачи и сделать 2-3 расчетных задания. а саелаю это!

9: Я, Дьячтов Вадим, из групз: 23581\_4, возможно скоро получу зачет по теории верЯятностей у Никитина Кирилла бячеславовица. Для этого потребуеося прорешать эти прПстейшие задачи и сделать 2-3 расчетных задания. Я саелаю это!

10: Я, Дьячтов Вадим, из групп: А3501\_4, возможно скоро полу№у зачет по теории вероятностей у Никитина Кирилла Вяяеславовича. Аля этого потребуеося прорешать эти простейшие задачи и сделать!2-3 расчетных задания. Я саелаю это!

11: Я, Дьячтов Вадим, из групп: 23501\_4, возможно скорЯ получу зачет по теории вероятностей у Никитина Кирилла Вячеславовича. Аля этого потребуется прорешать эти простейшие задачи и сделать 2-3 расчетных задания. Я сделаю это!

12: Я, Дьячтов Вадим, из групп: 23501\_4, возможно скоро получу зачет по теории вероятностей у Никитина Кирилла Вячеславовича. Аля этого потребуется прорешать эти простейшие задачи и сделать 2-3 расчетных задания. Я саелаю это!

13: Я, Дьячтов Вадим, из группы 23501\_4, возможно скоро получу зачет по теории вероятностей у Никитина Кирилла Вячеславовича. Аля этого потребуется прорешать эти простейшие задачи и сделать 2-3 расчетных задания. Я сделаю это!

14: Я, Дьячтов Вадим, из группы 23501\_4, возможно скоро получу зачет по теории вероятностей у Никитина Кирилла Вяяеславовича. Аля этого потребуется прорешать эти простейшие задачи и сделать 2-3 расчетных задания. Я сделаю это!

15: Я, Дьячтов Вадим, из группы 23501\_4, возможно скоро получу зачет по теории вероятностей у Никитина Кирилла Вячеславовича. Аля этого потребуется прорешать эти простейшие задачи и сделать 2-3 расчетных задания. Я сделаю это!

16: Я, Дьячтов Вадим, из группы 23501\_4, возможно скоро получу зачет по теории вероятностей у Никитина Кирилла Вяяеславовича. Аля этого потребуется прорешать эти простейшие задачи и сделать 2-3 расчетных задания. Я сделаю это!

17: Я, Дьячтов Вадим, из группы 23501\_4, возможно скоро получу зачет по теории вероятностей у Никитина Кирилла Вячеславовича. Для этого потребуется прорешать эти простейшие задачи и сделать 2-3 расчетных задания. Я сделаю это!

18: Я, Дьячтов Вадим, из группы 23501\_4, возможно скоро получу зачет по теории вероятностей у Никитина Кирилла Вячеславовича. Для этого потребуется прорешать эти простейшие задачи и сделать 2-3 расчетных задания. Я сделаю это!

}

\subsubsection{Вероятности букв задаются исходя из частоты встречания}

\subsection{Расчет энтропии и количества информации}

\section{Передача сообщения многократным дублированием}

\subsection{Определение переданного сообщения}

\subsection{Расчет энтропии и количества информации}

\section{Выводы}

\section*{Приложение}

\subsection*{Изменение распределения апостериорных вероятностей}

%\foreach \n in {1,...,18}{%
%\begin{figure}[H]
%\begin{center}
%	\vspace{-0.5cm}
%	\includegraphics[scale=0.61]{uniform\n}
%	%\caption{}
%	\vspace{-0.5cm}
%\end{center}
%\end{figure}
%}

\end{document}
